\documentclass[11pt,fleqn,a4paper]{article}
\usepackage{amsmath,amssymb}

\usepackage{natbib}
\usepackage[noend]{algpseudocode}
\usepackage{algorithm}
\renewcommand{\algorithmicrequire}{\textbf{Input:}}
\renewcommand{\algorithmicensure}{\textbf{Output:}}
\usepackage{mySweave}

\DeclareMathOperator*{\Lor}{\lor}
\DeclareMathOperator*{\Land}{\land}
\newcommand{\notodot}{\odot\kern-0.85em/\:\,}
\title{Algorithms of the editrules package}

\author{Mark van der Loo and Edwin de Jonge}

\begin{document}
\maketitle
Preparing data prior to statistical analysis can be a time-consuming
and costly task. At National Statistics Institutes (NSI's), such
preperatory steps (``data editing'') can consume up to 40\% of the resources
dedicated to a statistical publication \citep{waal:2011}. It is therefore highly desireble to
automate these processes as much as possible.

At the core of automated data editing, are so-called {\em edit rules} or {\em
edits}, in short, which represent record-wise restrictions that data must obey.
Examples include range restrictions, sum rules, excluded value combinations for
categorical data and the more complicated {\em conditional edits}. Conditional
edits are restrictions which are written in {\sf if-then-else} form.

We have recently released version {\sf 2.5} of our {\sf R} extension package
{\sf editrules}, (see \cite{loo:2012a} and references therein), which
implements automated rule-based data editing methods. The {\sf editrules}
package serves as a toolbox for rule manipulation, data checking and error
localization. In this paper we give details on some of the core algorithms
pertaining especially to the handling of conditional restrictions. This paper
supplements the papers describing how numerical \citep{jonge:2011}
or categorical \citep{loo:2012} edits are handled  by the package.



\clearpage
\bibliographystyle{chicago}
\bibliography{editrules}

\end{document}


\subsection{The {\sf editset} object}
\label{ss:editset}
A record in a mixed dataset can be described as an element of the domain
\begin{equation}
\label{eq:domain}
R = D_1\times D_2\times\cdots\times D_p\times \mathbb{R}^{q}.
\end{equation}
Here, $p$ and $q$ are positive integers that denote the number of categorical
and numerical variables respectively. The $D_k$ ($k\in1,2,\ldots,p$) denote the
possible categories for each categorical variable. Henceforth, we shall denote a
record as ${\bf r}=(v_1,v_2,\ldots,v_p,x_1,x_2,\ldots,x_q)=({\bf v},{\bf x})$, where
the $v_k$ are categorical and the $x_k$ are numerical variables.

Conditional edits can be formulated in several ways in the editrules package.
In mathematical notation, the most general form is given by
\begin{eqnarray}
\label{eqedit1}
\lefteqn{
\textrm{\sf if } {\bf v}\in F \Land {\bf Ax}\boldsymbol{\odot}{\bf b} \textrm{ \sf then {\sc false},}}\\
\textrm{where }\\
&&\begin{array}{l}
    F \subset D_1 \times D_2 \times\cdots\times D_p \textrm{ and}\\
{\bf A} \in \mathbb{R}^{m\times q} \textrm{, }  
    {\bf b} \in \mathbb{R}^{m}\textrm{, }   
    \boldsymbol{\odot}\in\{<,\leq,\geq, >\}^m. \nonumber
\end{array}
\end{eqnarray}
Here, it is understood that when both the categorical data is in a subdomain
$F$ {\em and} every linear inequation in the predicate is obeyed, then the record
is invalid. Observe that $F$ may be a set of categorical restrictions, since the
record-wise union of two categorical edits is also a categorical restriction. 

This formulation includes all types of conditional edits as discussed by
\cite{waal:2003}. However, since any condition in the predicate can be moved to
the consequent in negated form this formulation is slightly more general,
allowing numerical and categorical restrictions in both the predicate and the
consequent. The consequence of this flexibility is that that no linear
{\em equalities} are allowed in either predicate or consequent because negated
equalities are nonlinear.

In practice, one will define simpler edits, with one rule in the predicate and one
rule in the consequent, for example:
\begin{equation}
\textrm{\sf if } v\in e \textrm{ \sf then } {\bf a}\cdot{\bf x}\odot b,
\end{equation}
where $e$ is a categorical edit, {\bf a} is a real vector and $b$ is a
constant.  Here, the linear restriction in the consequent has to be obeyed for
a record to be {\em valid}. As an example of an edit with two numerical
restrictions, consider
\begin{equation}
\textrm{\sf if } {\bf a}\cdot {\bf x} \odot b \textrm{ \sf then } {\bf c}\cdot{\bf x} \odot' d.
\end{equation}
Here, the restriction in the consequent ($ {\bf c}\cdot{\bf x} \odot' d$) only
needs to be obeyed when ${\bf x}$ obeys the restriction in the predicate for a record to
be valid.


Internally, conditional edits are stored by assigning to each
linear condition $k$ a dummy variable $t_k$, which is defined as
\begin{equation}
\label{eqdummy}
t_k = \left\{\begin{array}{l}
\textrm{\sf true}\textrm{ when } {\bf a}_k{\bf \cdot x}\odot b_k\\
\textrm{\sf false}\textrm{ when }{\bf a}_k{\bf \cdot x}\notodot b_k,
\end{array}\right.
\end{equation}
where each ${\bf a}_k$ is a row of {\bf A} [Eq.\ \eqref{eqedit1}].
This way, the edits of Eq.\ \eqref{eqedit1} may be written as
\begin{equation}
\label{eqdummyedit1}
\textrm{\sf if } {\bf v} \in F \land t_1\land t_2\land\cdots\land t_m \textrm{ \sf then false},
\end{equation}
where $t_1\cdots t_m$ correspond to the $m$ linear restrictions of $\bf A$ in
Eq.\ \eqref{eqedit1}.  Equation \eqref{eqdummyedit1} can be stored in a boolean
representation as described in \cite{loo:2011b}. The {\sf editrules} package
uses an object class called {\sf editarray} to store such a representation.
The linear restrictions $\bf Ax \odot b$ for all conditional edits are gathered
in an {\sf editmatrix} object \citep{jonge:2011}. 

To store conditional edits, along with pure numerical and pure categorical
edits, we implemented the class {\sf editset}, which can be denoted as
\begin{equation}
\label{eqeditset}
E = \langle E_{\sf num}, E_{\sf mixcat}, E_{\sf mixnum};{\sf condition}\rangle.
\end{equation}
Here, $E_{\sf num}$ is an {\sf editmatrix} object holding unconditional numerical edits. 
$E_{\sf mixcat}$ is an {\sf editarray} object that stores ``pure categorical'' edits, as well as 
conditional edits containing the dummy variables $t_k$. 
The object $E_{\sf mixnum}$ is an {\sf editmatrix}, holding numerical edit terms occurring 
in conditional edits. The row names of $E_{\sf mixnum}$ correspond to dummy variables $t_k$ 
used in $E_{\sf mixcat}$. Finally, the
optional {\sf condition} attribute is an {\sf editmatrix} holding a
number of extra conditions (assumptions) under which the edits in $E_{\sf
num}$, $E_{\sf mixnum}$ and $E_{\sf mixcat}$ must hold. It will be discussed in
detail in Section \ref{ss:editlist}. Under normal circumstances ({\em i.e.} if only
editrules built-in functions are used to build and manipulate {\sf editsets})
either the {\sf condition} attribute or $E_{\sf mixnum}$ will be empty.

The functionality of {\sf editrules} is implemented such that the user can
remain agnostic of the internal representation of edit sets. There is however a
second representation of conditional edits which is used and returned by
several basic manipulation functions. This representation, stored in the form
of an {\sf editlist} or {\sf editenv} will be described in Section
\ref{ss:editlist} below.



%%%%%%%%%%%%%%%%%%%%%%%%%%%%%%%%%%%%%%%%%%%%%%%%%%%%%%%%%%%%%%%%%%%%%%%%%%%
\subsection{Untangling conditional edits}
\label{ss:editlist}
%
%
\begin{table}[t]
\caption{Edit separation functions. Each function accepts an {\sf editset} as input.}
\label{tab:separators}
\begin{tabular}{lp{0.8\textwidth}}
\hline
Function & Description\\
\hline
{\sf contains}  & Detects which edit contains which variable\\
{\sf plot}      & Plot the dependency graph\\
{\sf blocks}   & Splits an {\sf editset} in independent edits not sharing any variables\\
{\sf disjunct} & Splits an {\sf editset} in disjunct sets, not containing mixed edits\\
{\sf condition} & Returns the {\sf editmatrix} holding the conditions for an {\sf editset} generated by {\sf disjunct}\\
{\sf separate} & Uses {\sf blocks}, simplifies the results, and calls {\sf disjunct} on the remaining {\sf editset}s\\
\hline
\end{tabular}
\end{table}

As mentioned in the introduction, (conditional) edits are often interrelated by
shared variables, which complicates important manipulations like variable
elimination or value substitution. To make such manipulations more tractable
for conditional edits, the {\sf editrules} package has several methods to
analyse dependencies and to disentangle edit sets. An overview of available
functions is given in Table \ref{tab:separators}.
%
%\begin{figure}[t]
%<<fig=TRUE,echo=FALSE>>=
%par(oma=c(0,0,0,0),mar=c(0,0,1.5,0))
%set.seed(2)
%plot(E)
%@
%\caption{A graph view of the edits of Figure \ref{fig:editfile}, generated  with
%{\sf plot(E)}.  Squares represent edits, circles represent variables. An edge
%indicates that a variable occurs in an edit. The {\sf blocks} function
%separates an {\sf editset} into the independent editsets corresponding to the
%independent graphs shown here. Positioning of nodes is based on an optimization
%with random starting values and may therefore differ between repeated plots.
%}
%\label{fig:graph}
%\end{figure}
%

The function {\sf contains} returns a boolean matrix, indicating which
variables (columns) occur in which edit (rows). This connectivity matrix can be
represented as a graph, and in {\sf editrules}, the default {\sf plot} function
has been overloaded for {\sf editset} (and {\sf editmatrix} and {\sf
editarray}) to plot the dependency graph of a set of edits. The {\sf igraph0} package
of \cite{csardi:2006} is used to build the graphical representation. As an example,
the dependency graph of the edits in Figure \ref{fig:editfile} is plotted in
Figure \ref{fig:graph}.


The set of edits clearly consists of three unconnected blocks, not sharing any
variable.  The function {\sf blocks} separates an {\sf editset} into its
constituting independent edits and returns a {\sf list} of independent {\sf
editset} objects. The {\sf blocks} function has been introduced before in
\cite{jonge:2011} and \cite{loo:2011b} for {\sf editmatrix} and {\sf editarray}
objects respectively.

Besides the obvious separation based on the dependency graph, conditional edits
can be separated by evaluating all combinations of assumptions about the truth
values of restrictions in the predicates. In the editrules package, such a
separation is applied (partially) to disentangle numerical from categorical
restrictions. The result is a set of simpler {\sf editset}s, where each {\sf
editset} corresponds to a valid region of the domain $R$ [Eq.\
\eqref{eq:domain}].

To clarify this concept, we will work through the following example (we will
return to this example in section \ref{ss:elimination}). Consider the following
set of edits on the domain $(x,y)\in \mathbb{R}^2$.
%
\begin{equation}
G = \left\{\begin{array}{l}
\textrm{\sf if } x \geq 0 \textrm{ \sf then } y\geq 0\\
\textrm{\sf if } x \geq 0 \textrm{ \sf then } x \leq y\\
\textrm{\sf if } x < 0 \textrm{ \sf then } y < 0\\
\textrm{\sf if } x < 0 \textrm{ \sf then } x - y < -2. \\
\end{array}\right.
\label{eq:edits}
\end{equation}
These edits are all connected, since all of them contain both $x$ and $y$.
Figure \ref{figprojections} shows (in gray) the valid areas in the $x$-$y$ plane,
defined by these edits. From the sketch it is clear that the edits can be
separated in two purely linear edit sets, and that a record must obey the rules
defining either the left or the right gray area.
%
%\setkeys{Gin}{width=0.5\textwidth}
%\begin{figure}
%\centering
%<<fig=true,echo=false>>=
%    par(mar=rep(0,4),oma=rep(0,4))
%    lwd=2
%    pcol <- "#E3E3E3" 
%    x <- c(0,4)
%    plot(x,x,"l",axes=FALSE,xlab="",ylab="",xlim=c(-4,4),ylim=c(-2,3),lwd=lwd)
%    polygon(c(0,3.9,3.9),c(0,3.9,0),border=NA,col=pcol)
%    polygon(c(-2,-3.9,-3.9),c(0,0,-1.9),border=NA,col=pcol)
%    lines(x,x,lwd=lwd)
%    abline(v=0,lwd=lwd)
%    abline(h=0,lwd=lwd)
%    x2 <- c(-2,-4)
%    lines(x2,x2+2,lwd=lwd)
%    lines(c(0,5),c(0,0),lwd=5)
%    points(0,0,pch=16,cex=2)
%    lines(c(-2,-5),c(0,0),lwd=5)
%    text(4,-0.2,expression(x),cex=2)
%    text(-0.2,3,expression(y),cex=2)
%@
%\caption{Graphical representation (in gray) of the valid areas defined by the
%edits of Eq.\ \eqref{eq:edits}. Depicted are sections of the bordering lines
%$y=x$ and $y=x+2$.  The bold lines indicate the projections of the gray areas
%along the $y$-axis.
%}
%\label{figprojections}
%\end{figure}

Algebraically, this can be done by working out the assumptions $x\geq 0$ is
$\textrm{\sf true}$ ($\textrm{\sf false}$) and consequently $x<0$ is $\textrm{\sf false}$ ($\textrm{\sf true}$). We obtain
\begin{equation}
G = 
\left(
\textrm{if } x\geq0 \textrm{ \sf then }  \left\{\begin{array}{l}
        y\geq 0\\
        x\leq y\\
\end{array}\right.\right) \veebar
\left(
\textrm{if } x < 0 \textrm{ \sf then } \left\{\begin{array}{l}
        y < 0\\
        x - y < 2\\
\end{array}\right. 
\right).
\label{eq:editlist}
\end{equation}
The exclusive or, indicated by $\veebar$, stems from the fact that the not all
predicates in Eq.\ \eqref{eq:edits} can be simultaneously valid.
Geometrically, it reflects the notion that a record cannot be in the left and
right gray area at the same time. We will return to this example in the context
of variable elimination in Section \ref{ss:elimination}.



%
%
\begin{algorithm}[t]
\caption{Determine all feasible convex regions of an {\sf editset}.}
\label{alg:editlist}
\begin{algorithmic}[1]
    \State $S\leftarrow \varnothing$
    \Procedure{disjunct}{$E$}
    \State $T\leftarrow \{t : t \textrm{ is a dummy variable of }E \}$
    \If {$T = \varnothing$}
        \State $S \leftarrow S\cup E$
        \State {\bf return}
    \EndIf
        \State Choose a $t$ from $T$
        \State $E\leftarrow \textrm{\sf substValue}(E,t,\textrm{\sf true})$
        \If { ${\textrm{\sf isFeasible}\left(E_{\sf num}\cup \textrm{\sf condition}(E)\right)} \land \textrm{\sf isFeasible}(E_{\sf mixnum})$ }
\label{algp1}  
        \State $\textrm{\sc disjunct}(E)$
        \EndIf
        \State $E\leftarrow \textrm{\sf substValue}(E,t,false)$
        \If { ${\textrm{\sf isFeasible}\left(E_{\sf num}\cup \textrm{\sf condition}(E)\right)} \land \textrm{\sf isFeasible}(E_{\sf mixnum})$ }
\label{algp2} 
        %\If { $E_{\sf num}$, {\sf condition($E$)} and $E_{\sf mixnum}$ all feasible }
        \State $\textrm{\sc disjunct}(E)$
        \EndIf
    \EndProcedure
\end{algorithmic}
\end{algorithm}

The above example is generalized by the recursive procedure shown in Algorithm
\ref{alg:editlist}. The algorithm runs depth-first through a binary tree,
generating all possible assumptions about the truth values of conditional
numerical edits. Each leaf of the tree corresponds to one complete set of
assumptions about truth values of the linear restrictions occurring in
conditional edits.

Technically, assuming a truth value for a conditional numerical edit amounts to
substituting the value of a dummy variable in the $E_{\sf mixnum}$ part of an
{\sf editset}, copying the assumption to the {\sf condition}  attribute of the
{\sf editset}, and updating $E_{\sf mixcat}$ and $E_{\sf num}$ accordingly.
The {\sf substValue} function, which will be discussed in more detail in
Section \ref{sssubstitution}, is equipped for this. 

To save computational time, the binary tree is pruned whenever an inconsistent
set of edits arises, so no spurious edit sets are generated. The procedure uses
the function {\sf isFeasible} on objects of class {\sf editmatrix} and {\sf
editarray} which have been discussed in \cite{jonge:2011} and \cite{loo:2011b}
respectively.  Note that in the pruning conditions in lines \ref{algp1} and
\ref{algp2}, the feasibility of the union of $E_{\sf num}$ with the generated
numerical conditions is checked. This is done to make sure that no conditions
can arise that contradict the pure numerical edits. This can only happen when
the {\sf editset} contains internal contradictions, and implementing it this
way will help identifying infeasible sets of conditional edits, to be discussed
in Section \ref{ss:elimination}. Feasibility checks can be computationally
expensive, so some short-circuiting is done in the implementation of
lines \ref{algp1} and \ref{algp2} to avoid unnecessary computations.


The {\sf editrules} function {\sf disjunct} implements the procedure of
Algorithm~\ref{alg:editlist}. The name ``disjunct'' derives from the fact that a
valid record is in precisely one of the valid regions defined by the resulting
set of {\sf editset} objects.  Every editset gains an extra attribute called
{\sf condition}, which holds the predicates  pertaining to that {\sf editset}.
As an example, consider the output of calling {\sf disjunct} on the edits defined in
Eq.\ \eqref{eq:edits}, and compare with Eq.\ \eqref{eq:editlist}.
%<<echo=false>>=
%G <- editset(expression(
%    if ( x >= 0 ) y >= 0,
%    if ( x >= 0 ) x <= y,
%    if ( x < 0 ) y < 0,
%    if ( x < 0 ) x - y < -2
%))
%@
%<<>>=
%disjunct(G)
%@
The function {\sf condition} retrieves the predicate restrictions of an {\sf editset},
in the form of an {\sf editmatrix}.

By default the {\sf disjunct} function returns all {\sf editset}s as a {\sf
list}. However, operations on lists can be time-consuming when elements in a
list have to be treated and replaced one by one. For this reason, {\sf
disjunct} accepts an optional {\sf type} argument.  If {\sf type="env"}, the
set of {\sf editsets} are returned as an environment, which may be used to
avoid unnecessary copying. To allow overloading of functions such as {\sf
eliminate} and {\sf localizeErrors}, the result of {\sf disjunct} is of class
{\sf editlist} or {\sf editenv}. 

%\begin{figure}
%\setkeys{Gin}{width=0.9\textwidth}
%<<fig=true,echo=false>>=
%v <- disjunct(E)
%par(oma=c(0,0,0,0),mar=c(0,0,1.5,0),mfrow=c(2,2),pty='m')
%for ( i in 1:length(v) ){
%    set.seed(1)
%    plot(
%        v[[i]],
%        main=paste(as.character(editrules:::condition(v[[i]])),collapse=', ')
%    )
%}
%@
%\caption{Connectivity graphs of the disjunct edit sets
%generated from the edits of Figure \ref{fig:editfile}. There are no paths from
%numerical variables ($x$, $y$, $z$, $u$, $v$, $w$) to categorical 
%variables ($A$, $B$, $C$, $D$) anymore. The titles of the subplots indicate
%the predicates for each {\sf editset}.}
%\label{figdisjunct}
%\end{figure}
%
%
As an example of how {\sf disjunct} separates numerical from categorical
variables, the four editsets, resulting from passing the edits of Figure
\ref{fig:editfile} to {\sf disjunct} are plotted in Figure \ref{figdisjunct}.
The original editset has three dummy variables, corresponding to $x>0$, $y>0$,
and $x>y$, yielding eight disjunct spaces. Four of these are empty, because not
all combinations of truth values are valid. For example, the conditions
$x>y=\textrm{\sf true}$, $x>0=\textrm{\sf false}$ and $y>0=\textrm{\sf  true}$ contradict.  Note that contrary to
Figure \ref{fig:graph}, there are no paths connecting numerical variables
($x$,$y$,$z$,$u$,$v$,$w$) to categorical variables ($A$, $B$, $C$, $D$) in any
of the dependency graphs.

Finally, we note the {\sf separate} function that combines {\sf blocks} and
{\sf disjunct}. Function {\sf separate} blocks an editset based on variable
occurrence and simplifies the blocks to objects of class {\sf editmatrix} or
{\sf editarray} where possible. After this, the remaining {\sf editset} objects
are split into an {\sf editlist} using {\sf disjunct}. The result is returned
as an {\sf R} {\sf list}. 

\subsection{Value substitution}
Recall that conditional edits in an {\sf editset} are stored as a combination of
categorical edits on dummy variables pertaining to linear edits in $E_{\sf mixnum}$.
Purely numerical edits are stored in a separate subobject called $E_{\sf num}$.
To substitute a value in an {\sf editset}, four cases are distinguished.

The first two cases are the simplest and concern variables which are either
numerical variables not occurring in $E_{\sf mixnum}$ or categorical variables. These
cases are handled by dispatching the appropriate {\sf substValue} methods for
$E_{\sf num}$ or $E_{\sf mixcat}$ ({\sf editmatrix} and {\sf editarray}
respectively) which have been described in \cite{jonge:2011} and
\cite{loo:2011b}.

The third case occurs when the substituted variable is a numerical variable,
occurring in a conditional edit. In that case
the following actions are performed.
%
\begin{enumerate}
\item If the variable occurs in $E_{\sf num}$ substitute its value there.
\item Substitute the variable in $E_{\sf mixnum}$.
\item If any edits in $E_{\sf mixnum}$ have become obviously redundant ({\em e.g.} $0 < 1$) or
obvious contradictions ({\em e.g.} $1 < 0$), substitute respectively \textrm{\sf true} or \textrm{\sf false} for
the corresponding dummy variables in $E_{\sf mixcat}$.
\item Remove all edits of the form $\textrm{\sf if } \textrm{\sf true} \textrm{ \sf then  } t$
from $E_{\sf mixcat}$. (Recall that $t$ is a dummy variable, see Eqs.\ \eqref{eqdummy}-\eqref{eqdummyedit1}).
\item If the dummy variables $t$ of step 4 do not occur anywhere in $E_{\sf
mixcat}$ anymore, move the corresponding linear inequality from $E_{\sf mixnum}$ to $E_{\sf num}$.
\item Remove all edits of the form $\textrm{\sf if } \textrm{\sf false} \textrm{ \sf then  } t$.

\end{enumerate}
%
%
As an example, consider the code below.
%<<>>=
%F <- editset(expression(
%x + y == z,
%if ( x > 0 ) y > 0))
%substValue(F,"x",3)
%@
%Here, substituting $x=3$, yields $\textrm{\sf  if } \textrm{\sf true} \textrm{ \sf  then }
%y>0$, so the consequent $y>0$ may be added to the numerical edits (here:
%$x+y=z$).  In case we substitute $x$ with $-3$ we obtain $\textrm{\sf  if } -3
%> 0 \textrm{ \sf  then } y>0$, yielding a false predicate and the conditional
% edit can be deleted:
%<<>>=
%substValue(F,"x",-3)
%@

The fourth and final case concerns substitution of dummy variables, which was
discussed in Section \ref{ss:editlist}.  Recall that dummy variables are used
to code linear inequalities that occur in conditional edits. If a dummy
variable is substituted, {\sf substValue} moves the corresponding numerical
edit from $E_{\sf mixnum}$ to the {\sf condition} attribute of the {\sf
editset}. Next, $E_{\sf mixcat}$ is updated accordingly.  Substituting dummy
variables explicitly is not normally done by end users, as the names and values
of dummy variables remain hidden to them.  It is used internally by {\sf
disjunct}. 



\subsection{Acceleration by data model checking}
Observe that the number of nodes of the branch-and-bound tree grows as $2^n$,
with $n$ the number of variables (and hence the depth of the tree).  In some
cases the tree depth can be reduced by noting that variables which violate
simple, single-variable range edits will always be part of the error
localization solution. These single-variable edits are easily checked (in a
vectorized manner) and variables violating such edits can be removed from the
multivariate localization problem. 

The solution to an error localization problem for a number of records in a
dataset with $n$ variables can be represented as matrix ${\bf M}$ where
$M_{ij}>0$ if variable $j$ in record $i$ must be changed and $0$ otherwise.  To
construct such a matrix, we assume a set of single-variable edits, and
proceed as follows.  Define ${\bf V}$ as the matrix where $V_{ik}=1$ if
record $i$ violates edit $k$ and $0$ otherwise.  To account for missing values,
we also set $V_{ik}=1$ when edit violation cannot be established because the
value of the variable to which range edit $k$ pertains is missing.  Also,
define ${\bf C}$ as the matrix with $C_{kj}=1$ if edit $k$ contains
variable $j$ and $0$ otherwise. The matrix ${\bf M}$ is then given by
\begin{equation}
{\bf M} = {\bf V}{\bf C}.
\end{equation} 

In {\sf editrules}, this operation has been implemented in the function {\sf checkDatamodel}.
This function returns an object of class {\sf errorLocation}, but does not yield a full solution,
as demonstrated by the example of Figure \ref{R:checkdatamodel}.

%\addtocounter{figure}{1}
%\begin{Rcode}
%<<>>=
%E <- editmatrix(expression(
%    x > 0,
%    x + y == 2
%))
%(dat <- data.frame(x=c(-1,2),y=c(3,0)))
%checkDatamodel(E,dat)$adapt
%localizeErrors(E,dat)$adapt
%@
%\caption{An example of {\sf checkDatamodel}, which only detects violation of range edits,
%while {\sf localizeErrors} finds the full solution. The result of both functions is an
%object of class {\sf errorLocation} which contains (amongst other things) for each record
%which variables should be adapted. Clearly, in the first record $(x=-1,y=3)$ both $x$ and $y$
%must be adapted. However, {\sf checkDatamodel} only checks the one-dimensional edits, {\em i.c.}
%the edit $x>0$.
%}
%\label{R:checkdatamodel}
%\end{Rcode}



\end{document}


